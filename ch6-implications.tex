\chapter{Conservation Implications: Why should we care about detectability?}

\par Even though imperfect detectability is often the norm when surveying for birds and other organisms, detectability is often overlooked in models used for conservation purposes \citep{bennett_how_2024}.
This oversight is not limited to small studies either, as long-running, large-scale surveying programs such as the North American Breeding Bird Survey also fail to account for detectability in modelling.
In this short chapter, I will synthesize some of the conservation implications of systematically estimating detectability in landbirds, particularly as they related to the methods outlined in chapters 3 - 5. 
This chapter was inspired by a plenary session given by Professor Gerardo Ceballos at the 2022 Ecological Society of American/Canadian Society for Ecology and Evolution joint conference in Montreal, Quebec, Canada.
In the last 15 minutes of his plenary, he discussed his advice for upcoming and current graduate students in the field of ecology, evolution, conservation science, and other related fields.
His one piece of advice (to paraphrase him) was to ``make sure that every single piece of research you do has a conservation implication to it", and to ``explicitly make a `conservation implications' section of every single paper that you write".
Given this thesis is comprised of several related research papers, it seemed appropriate to synthesize the conservation implications of these projects together in one chapter.

\section{Correcting for Detectability in Long-term Studies}

\par North American Breeding Bird Survey data have been highly useful in conservation \citep{hudson_role_2017}.
The dataset is long term, with close to 60 years of longitudinal data; it is highly structured, meaning there are specific protocols that must be followed in order to maintain a sense of homogeneity between counts; and the routes span a vast area of Canada and the US.

\par Despite being a highly-structured dataset that has many biases accounted for, observed counts on a BBS route will still suffer from observational biases based on changing landcover through time at survey locations.
By explicitly incorporating detectability into the BBS analyses, we can correct for some of these additional sources of bias in the dataset in our attempts at trying to model the true underlying abundance of birds.
Additionally, it is well-known that landscapes for any given route within the BBS will have likely changed over the potentially 50+ years that the route has been run \citep{sauer_first_2017}.
For example, increased agricultural and housing needs in some areas have come at the expense of forest cover, and some of the roads used by the BBS have become larger and busier. 
I have shown here how the NA-POPS detectability estimates, in combination with time-series of long-term habitat changes at BBS point locations, can be used to generate detectability offsets to adjust for landscape alterations over time. 

\par Given detectability is currently a dynamic field of research, future BBS outputs that include the most up-to-date research on detectability can begin to generate trends that more closely reflect the true increases or decreases of bird species.
This is particularly important in jurisdictions such as Canada that rely on estimates from the BBS to inform the listing and delisting process of species at risk.
Species where the true trend is masked by changes in detectability may not get properly listed, and thus could face continuing threats and issues that come with ``monitoring to extinction" \citep{martin_acting_2012, lindenmayer_counting_2013}.
By including this key piece of information, and by being able to account for a large source of bias, the BBS can continue to provide excellent information on status, trends, and relative abundance of birds in North America well into the future.

\section{Improving Population Size Estimates}

\par Detectability offsets derived from NA-POPS can be used to improve population estimates of landbirds in North America. 
Because they include separate estimates for on- vs. off-road surveys, the NA-POPS offsets would represent an improvement over the current PIF population estimates which assume that the roadside BBS is representative of the entire landscape \citep{rosenberg_partners_2016, stanton_estimating_2019}, even though it is well known that detectability of birds changes with on- vs off-road surveys \citep{sauer_first_2017, yip_sound_2017, cooke_road_2020}, and estimates of bird populations have been shown to improve when accounting for roadside bias \citep{solymos_lessons_2020}. 
Generating population sizes using EDRs that account for roadside status and forest coverage may improve the accuracy of continental estimates of population for many landbird species. 
Additionally, integrating more refined information on density would have the simultaneous benefits of accounting for potential variations in detectability, reducing biases within and among monitoring programs, and generating useful information on local population sizes of birds that could inform conservation prioritizations \citep{veloz_improving_2015}.

Additionally, the hierarchical Bayesian models in Chapter 4 provide an excellent opportunity for detection probabilities to be estimated with higher precision for several species of landbirds in North America, and also provides a method to predict detection probabilities for species which have very little (or even no) data. 
This means that conservation problems that rely on estimates of detectability such as Value of Information analyses \citep{canessa_when_2015, bennett_when_2018} or prioritizations \citep{hanson_prioritizr_2022} can make more effective decisions.
For organizations such as Partners in Flight, Boreal Avian Modelling project \citep{cumming_toward_2010}, National Audubon Society, and others that explicitly rely on detectability information for bird population size estimates or abundance modelling using small point count datasets, these improved estimates (and new estimates and predictions) provide a way forward for continuing to produce accurate estimates of population size that are based on data-driven detectability estimates.
For applications that tend to rely on known presences, such as threat assessments, our method opens new possibilities for incorporating detectability, and thus gaining a fuller understanding of  potential locations of undetected occurrences, and more effective allocation of survey effort.

\section{Data Integration}

\par Data integration is a growing topic throughout ecology \citep{isaac_data_2020, miller_recent_2019, pacifici_integrating_2017, boersch-supan_integrating_2021}, especially as the number of data sets continue to grow \citep{binley_minimizing_2023}.
As such, recognizing what biases exist---in both professionally-collected data and community science data \citep{binley_data_2023}---across complementary data sets and working out methodologies to jointly model these data sets can improve inference of the state of a biological system.

\par Any single program has gaps in coverage that may bias the estimates. 
For example, the BBS data have provided the basis for estimates of trends in relative abundance for North American landbirds, but there are known biases in the sampling framework that cannot be filled using the BBS’s field methods \citep{thogmartin_sensitivity_2010, solymos_lessons_2020, us_geological_survey_strategic_2020}. 
As a roadside survey, the BBS has excellent coverage in areas where there are roads, such as the eastern United States, and poor coverage where there are few roads, such as the north (boreal and arctic regions of Canada and Alaska), Mexico, and alpine regions. 
Possibilities exist to fill these gaps by taking advantage of data available through other existing monitoring programs. 
For example, the IMBCR program collects data from montane and grassland regions in western and central USA \citep{pavlacky_statistically_2017}, and the Avian Knowledge Network \citep{iliff_avian_2009} and eBird data \citep{sullivan_ebird_2014} can be used to fill in gaps throughout the continent. 
Additionally, the PROALAS program has good coverage in Mexico \citep{ruiz_gutierrez_proalas_2020}, which could allow for better estimates of southern North American birds. 
Integrating these data into a single modelling framework could fill spatial gaps, address limitations, and complement BBS data and analyses \citep{miller_recent_2019, isaac_data_2020}.

\par NA-POPS estimates can be used for data integration across variations in survey duration and timing during the day and season, as well as observation conditions such as forest coverage and roadside-status.
Integrating information across disparate field programs and sampling protocols remains a key challenge to analyzing compilations of heterogeneous survey data, because the observed counts of birds during any particular survey do not provide comparable estimates of the true density of birds.
For example, the BBS conducts 3-minute, 400 m, roadside point count surveys, whereas the IMBCR program conducts mostly off-road, 6-minute, unlimited distance point count surveys and records detection distances.
However, using the QPAD offsets produced by NA-POPS, we can transform raw, survey-level counts into estimates of true density and account for differences among survey method and conditions \citep{stralberg_projecting_2015, solymos_lessons_2020}.
For example, BBS counts for a given species can be adjusted using QPAD offsets from NA-POPS for a 3-minute, 400-metre-radius count on a roadside, so that the BBS counts can be integrated with 6-minute, unlimited-distance offroad counts from IMBCR.
This can allow us to include disparate data sets in the same model, so we no longer have to make broad-scale inferences from a single survey (such as status and trends of North American birds derived solely from the BBS). Instead, we can begin deriving broad-scale inferences with broad-scale information via multiple surveys. 

\par Additionally, these QPAD offsets can then also be applied to semi-structured citizen science data that come from eBird \citep{sullivan_ebird_2014}, if we are able to filter and derive checklists that meet a stationary count protocol for a reasonably short period of time.
Several promising studies have demonstrated the utility of community science programs such as eBird in filling spatial gaps for abundance and species distribution models \citep{pacifici_integrating_2017, robinson_integrating_2020, joseph_data_2021}.
With data from eBird being available for researchers to download, future studies could consider generating roadside status and forest coverage variables for stationary protocol checklists which are reasonably short in length (e.g., <10 minutes).
Using GIS software, roadside status and forest coverage variables could be derived and, along with the temporal information from the checklist, can be used as input for the detectability functions in order to produce the estimate of density for that checklist. 


The point-level model introduced in Chapter 5 combined with the NA-POPS estimates provide a potentially simple way of integrating these data within the current BBS status and trend models, because the NA-POPS detectability estimates correct for biases in survey duration, survey radius, environmental conditions, and temporal conditions.
As such, once QPAD \citep{solymos_calibrating_2013} estimates are applied to a point count, including points from the BBS, it is effectively ``speaking the same language" as other point count data sets with QPAD offsets applied.
In this way, the data can be jointly considered under the same modelling framework.
We propose that future studies investigate how to directly include data from Boreal Avian Modelling project, Integrated Monitoring in Bird Conservation Regions, eBird, and other data sets across the continent, into the BBS modelling framework.
We suspect that such a model would have a similar model structure as Equation \ref{bbs_varprop}, but perhaps with extra terms for project-specific effects.
Because counts within a stratum are averaged for trend, even if additional data is sparse in a specific stratum, it will still contribute to the trend.

\section{Facilitating Open Science and Future Detectability Research}

\par NA-POPS is an open-access database, with several avenues for researchers to be able to explore and access these results. The unprocessed results can simply be downloaded from the GitHub Organization (https://github.com/na-pops/results), the summarized results and predictions can be visualized using the NA-POPS dashboard (found at https://na-pops.org), and the processed results can be accessed using the R package \ttt{napops} \citep{edwards_napops_2024}. Ease of use of these detectability functions will allow researchers from across North America to use these estimates where they see fit, scrutinize these estimates where there is disagreement, and explore deeper into species-specific estimates that are surprising or counterintuitive.

\par We also hope that this broad-scale synthesis of detectability estimates will inspire future work in landbird detectability across North America, as well as on a global scale. We have mentioned here several surprising findings, including some unexpected results concerning off-road vs on-road surveys. Additionally, we have highlighted several future avenues for more specific detectability research, including investigating spatial effects on cue rate and/or EDR, investigating potential observer effects on EDR, and the need for additional data from several geographic regions of North America. Because detectability is an important consideration in several modelling exercises and carries several conservation implications with it, we recommend that researchers wanting to run bird surveys strategically design their surveys such that the survey protocols allow for detectability estimates to be derived (i.e., \citet{matsuoka_reviving_2014}). 

\par Finally, we see this work as a proof-of-concept for use with other taxa.
Obviously, given the vast amount of bird data available, it was relatively easy to obtain estimates for enough data-rich species such that the few data-poor species could have enough information to borrow. 
However, one benefit that we have shown here is that Bayesian models can still derive reasonable predictions if there are informed enough priors.
For taxa such as butterflies for which some species have reasonable data and most have very little data \citep{lewthwaite_geographical_2022}, a researcher may still be able to glean reasonable estimates of detectability with the little data available, as long as detectability for some similar species has been examined.
We suggest that this type of model---where information about detectability is shared among species---be tested with taxa that generally have fewer data collected, and we encourage researchers to continue to make use of existing sources of data \citep{binley_minimizing_2023} to inform these models.

