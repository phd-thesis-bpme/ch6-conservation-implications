\chapter{Conservation Implications, Future Directions, and Closing Remarks}

\par Even though imperfect detectability is often the norm when surveying for birds and other organisms, detectability is often overlooked in models used for conservation purposes \citep{bennett_how_2024}.
This oversight is not limited to small studies either, as long-running, large-scale surveying programs such as the North American Breeding Bird Survey also fail to account for detectability in modelling.
In this final chapter, I will synthesize some of the conservation implications of systematically estimating detectability in landbirds, particularly as they related to the methods outlined in Chapters 3 - 5. 
This chapter was inspired by a plenary session given by Professor Gerardo Ceballos at the 2022 Ecological Society of American/Canadian Society for Ecology and Evolution joint conference in Montreal, Quebec, Canada.
In the last 15 minutes of his plenary, he discussed his advice for upcoming and current graduate students in the field of ecology, evolution, conservation science, and other related fields.
His one piece of advice (to paraphrase him) was to ``make sure that every single piece of research you do has a conservation implication to it", and to ``explicitly make a `conservation implications' section of every single paper that you write".
Given this thesis is comprised of several related research papers, it seemed appropriate to synthesize the conservation implications of these projects together in one chapter.
I will then end the chapter with some future directions and questions that I believe should be pursued in the realm of detectability research.
I note that these future directions will mostly relate to detectability in landbirds, but I do believe some of the technology and modelling work needing to be done will have implications further afield.

\section{Conservation Implications}

\subsection{Correcting for Detectability in Long-term Studies}

\par North American Breeding Bird Survey data have been highly useful in conservation \citep{hudson_role_2017}.
The dataset is long term, with close to 60 years of longitudinal data; it is highly structured, meaning there are specific protocols that must be followed in order to maintain a sense of homogeneity between counts; and the routes span a vast area of Canada and the US.

\par Despite being a highly-structured dataset that has many biases accounted for, observed counts on a BBS route will still suffer from observational biases based on changing landcover through time at survey locations.
By explicitly incorporating detectability into the BBS analyses, we can correct for some of these additional sources of bias in the dataset in our attempts at trying to model the true underlying abundance of birds.
Additionally, it is well-known that landscapes for any given route within the BBS will have likely changed over the potentially 50+ years that the route has been run \citep{sauer_first_2017}.
For example, increased agricultural and housing needs in some areas have come at the expense of forest cover, and some of the roads used by the BBS have become larger and busier. 
I have shown here how the NA-POPS detectability estimates, in combination with time-series of long-term habitat changes at BBS point locations, can be used to generate detectability offsets to adjust for landscape alterations over time. 

\par Given detectability is currently a dynamic field of research, future BBS outputs that include the most up-to-date research on detectability can begin to generate trends that more closely reflect the true increases or decreases of bird species.
This is particularly important in jurisdictions such as Canada that rely on estimates from the BBS to inform the listing and delisting process of species at risk.
Species where the true trend is masked by changes in detectability may not get properly listed, and thus could face continuing threats and issues that come with ``monitoring to extinction" \citep{martin_acting_2012, lindenmayer_counting_2013}.
By including this key piece of information, and by being able to account for a large source of bias, the BBS can continue to provide excellent information on status, trends, and relative abundance of birds in North America well into the future.

\subsection{Improving Population Size Estimates}

\par Detectability offsets derived from NA-POPS can be used to improve population estimates of landbirds in North America. 
Because they include separate estimates for on- vs. off-road surveys, the NA-POPS offsets would represent an improvement over the current PIF population estimates which assume that the roadside BBS is representative of the entire landscape \citep{rosenberg_partners_2016, stanton_estimating_2019}, even though it is well known that detectability of birds changes with on- vs off-road surveys \citep{sauer_first_2017, yip_sound_2017, cooke_road_2020}, and estimates of bird populations have been shown to improve when accounting for roadside bias \citep{solymos_lessons_2020}. 
Generating population sizes using EDRs that account for roadside status and forest coverage may improve the accuracy of continental estimates of population for many landbird species. 
Additionally, integrating more refined information on density would have the simultaneous benefits of accounting for potential variations in detectability, reducing biases within and among monitoring programs, and generating useful information on local population sizes of birds that could inform conservation prioritizations \citep{veloz_improving_2015}.

Additionally, the hierarchical Bayesian models in Chapter 4 provide an excellent opportunity for detection probabilities to be estimated with higher precision for several species of landbirds in North America, and also provides a method to predict detection probabilities for species which have very little (or even no) data. 
This means that conservation problems that rely on estimates of detectability such as Value of Information analyses \citep{canessa_when_2015, bennett_when_2018} or prioritizations \citep{hanson_systematic_2024} can make more effective decisions.
For organizations such as Partners in Flight, Boreal Avian Modelling project \citep{cumming_toward_2010}, National Audubon Society, and others that explicitly rely on detectability information for bird population size estimates or abundance modelling using small point count datasets, these improved estimates (and new estimates and predictions) provide a way forward for continuing to produce accurate estimates of population size that are based on data-driven detectability estimates.
For applications that tend to rely on known presences, such as threat assessments, our method opens new possibilities for incorporating detectability, and thus gaining a fuller understanding of  potential locations of undetected occurrences, and more effective allocation of survey effort.

\subsection{Data Integration}

\par Data integration is a growing topic throughout ecology \citep{isaac_data_2020, miller_recent_2019, pacifici_integrating_2017, boersch-supan_integrating_2021}, especially as the number of data sets continue to grow \citep{binley_minimizing_2023}.
As such, recognizing what biases exist---in both professionally-collected data and community science data \citep{binley_data_2023}---across complementary data sets and working out methodologies to jointly model these data sets can improve inference of the state of a biological system.

\par Any single program has gaps in coverage that may bias the estimates. 
For example, the BBS data have provided the basis for estimates of trends in relative abundance for North American landbirds, but there are known biases in the sampling framework that cannot be filled using the BBS’s field methods \citep{thogmartin_sensitivity_2010, solymos_lessons_2020, us_geological_survey_strategic_2020}. 
As a roadside survey, the BBS has excellent coverage in areas where there are roads, such as the eastern United States, and poor coverage where there are few roads, such as the north (boreal and arctic regions of Canada and Alaska), Mexico, and alpine regions. 
Possibilities exist to fill these gaps by taking advantage of data available through other existing monitoring programs. 
For example, the IMBCR program collects data from montane and grassland regions in western and central USA \citep{pavlacky_statistically_2017}, and the Avian Knowledge Network \citep{iliff_avian_2009} and eBird data \citep{sullivan_ebird_2014} can be used to fill in gaps throughout the continent. 
Additionally, the PROALAS program has good coverage in Mexico \citep{ruiz_gutierrez_proalas_2020}, which could allow for better estimates of southern North American birds. 
Integrating these data into a single modelling framework could fill spatial gaps, address limitations, and complement BBS data and analyses \citep{miller_recent_2019, isaac_data_2020}.

\par NA-POPS estimates can be used for data integration across variations in survey duration and timing during the day and season, as well as observation conditions such as forest coverage and roadside-status.
Integrating information across disparate field programs and sampling protocols remains a key challenge to analyzing compilations of heterogeneous survey data, because the observed counts of birds during any particular survey do not provide comparable estimates of the true density of birds.
For example, the BBS conducts 3-minute, 400 m, roadside point count surveys, whereas the IMBCR program conducts mostly off-road, 6-minute, unlimited distance point count surveys and records detection distances.
However, using the QPAD offsets produced by NA-POPS, we can transform raw, survey-level counts into estimates of true density and account for differences among survey method and conditions \citep{stralberg_projecting_2015, solymos_lessons_2020}.
For example, BBS counts for a given species can be adjusted using QPAD offsets from NA-POPS for a 3-minute, 400-metre-radius count on a roadside, so that the BBS counts can be integrated with 6-minute, unlimited-distance offroad counts from IMBCR.
This can allow us to include disparate data sets in the same model, so we no longer have to make broad-scale inferences from a single survey (such as status and trends of North American birds derived solely from the BBS). Instead, we can begin deriving broad-scale inferences with broad-scale information via multiple surveys. 

\par Additionally, these QPAD offsets can then also be applied to semi-structured citizen science data that come from eBird \citep{sullivan_ebird_2014}, if we are able to filter and derive checklists that meet a stationary count protocol for a reasonably short period of time.
Several promising studies have demonstrated the utility of community science programs such as eBird in filling spatial gaps for abundance and species distribution models \citep{pacifici_integrating_2017, robinson_integrating_2020, joseph_data_2021}.
With data from eBird being available for researchers to download, future studies could consider generating roadside status and forest coverage variables for stationary protocol checklists which are reasonably short in length (e.g., $< 10$ minutes).
Using GIS software, roadside status and forest coverage variables could be derived and, along with the temporal information from the checklist, can be used as input for the detectability functions in order to produce the estimate of density for that checklist. 

The point-level model introduced in Chapter 5 combined with the NA-POPS estimates provide a potentially simple way of integrating these data within the current BBS status and trend models, because the NA-POPS detectability estimates correct for biases in survey duration, survey radius, environmental conditions, and temporal conditions.
As such, once QPAD \citep{solymos_calibrating_2013} estimates are applied to a point count, including points from the BBS, it is effectively ``speaking the same language" as other point count data sets with QPAD offsets applied.
In this way, the data can be jointly considered under the same modelling framework.
I propose that future studies investigate how to directly include data from Boreal Avian Modelling project, Integrated Monitoring in Bird Conservation Regions, eBird, and other data sets across the continent, into the BBS modelling framework.
I suspect that such a model would have a similar model structure as Equation \ref{bbs_varprop}, but perhaps with extra terms for project-specific effects.
Because counts within a stratum are averaged for trend, even if additional data is sparse in a specific stratum, it will still contribute to the trend.

\subsection{Facilitating Open Science and Future Detectability Research}

\par NA-POPS is an open-access database, with several avenues for researchers to be able to explore and access these results. The unprocessed results can simply be downloaded from the GitHub Organization (https://github.com/na-pops/results), the summarized results and predictions can be visualized using the NA-POPS dashboard (found at https://na-pops.org), and the processed results can be accessed using the R package \ttt{napops} \citep{edwards_napops_2024}. Ease of use of these detectability functions will allow researchers from across North America to use these estimates where they see fit, scrutinize these estimates where there is disagreement, and explore deeper into species-specific estimates that are surprising or counterintuitive.

\par I also hope that this broad-scale synthesis of detectability estimates will inspire future work in landbird detectability across North America, as well as on a global scale. I have mentioned here several surprising findings, including some unexpected results concerning off-road vs on-road surveys. Additionally, I have highlighted several future avenues for more specific detectability research, including investigating spatial effects on cue rate and/or EDR, investigating potential observer effects on EDR, and the need for additional data from several geographic regions of North America. Because detectability is an important consideration in several modelling exercises and carries several conservation implications with it, I recommend that researchers wanting to run bird surveys strategically design their surveys such that the survey protocols allow for detectability estimates to be derived (i.e., \citet{matsuoka_reviving_2014}). 

\par Finally, I see this work as a proof-of-concept for use with other taxa.
Obviously, given the vast amount of bird data available, it was relatively easy to obtain estimates for enough data-rich species such that the few data-poor species could have enough information to borrow. 
However, one benefit that I have shown here is that Bayesian models can still derive reasonable predictions if there are informed enough priors.
For taxa such as butterflies for which some species have reasonable data and most have very little data \citep{lewthwaite_geographical_2022}, a researcher may still be able to glean reasonable estimates of detectability with the little data available, as long as detectability for some similar species has been examined.
I suggest that this type of model---where information about detectability is shared among species---be tested with taxa that generally have fewer data collected, and I encourage researchers to continue to make use of existing sources of data \citep{binley_minimizing_2023} to inform these models.

\section{Four Questions for Future Detectability Research}

\par This section will explore some of the questions or directions that I feel future statistical ecologists, quantitative biologists, or anyone with an interest in detectability, should explore.
These are by no means a comprehensive list.
Instead, they moreso seek to synthesize some of the common themes that have cropped up through my time working with other statisticians, biologists, and ecologists working on similar problems.
The recent past of computational advances, data availability, and code sharing has made it much more easier to begin to tackle these questions at scale, and so I look forward to working with these future biolgists, ecologists, and statisticians on these problems for hopefully the next few decades.

\subsection{Does space matter?}

\par How much spatial information do we need to actually account for when modelling broad-scale estimates of detectability in landbirds?
I think the answer to this question will be highly species-dependent, and I believe it will change depending on whether we are interested in changes in availability, perceptibility, or both.
Wide-ranging species such as Song Sparrow, American Robin, Black-capped Chickadee, etc., will likely experience differences in availability throughout the year depending on the survey location.
In particular with wide-ranging species, this difference in availability could either be on a north-south gradient (i.e., affected by daylight and temperature) or on an east-west gradient (i.e., affected by temperature and differences in ``green-up" times).
One coarse way to answer this question is to simply model availability as a function of a spatially-explicit unit such as Bird Conservation Region (BCR).
Doing this would allow potential variation to be captured by these discrete spatial units.
One further advantage to this is to model changes in \textit{perceptibility} by space as well.
Recall from Chapter 5 that, at least for Ovenbird, there appeared to be unaccounted-for biases in the perceptibility varprop parameters that may imply biases related to differences in forest coverage.
As previously noted, detectability can vary with forest type, not just amount of forest.
Because BCRs explicitly account for ecoregion and associated environmental variables, BCRs could act as a reasonable proxy for differences in vegetation, and therefore capture some of the extra variance in perceptibility.

\par One caveat to answering these questions is that one likely needs a lot of data.
The NA-POPS project has done a reasonable job at collating a large amount of data across a large spatial extent, and so there are likely enough species within that database to start answering this question.
However, one could consider also the use of hierarchical models to share information in space, such that data-sparse regions can borrow information from data-rich regions.
To take it a step further, this hierarchical model could also take on a more spatially-explicit form by only sharing information with nearest neighbours via ICAR models as in \citet{smith_spatially_2023}.
For particularly data-rich species, one could consider higher resolution spatial units, such as degree blocks, or directly include space in a model as latitude and longitude.

\subsection{Are availability and perceptibility truly independent?}

\par In this thesis, and typically in many studies of detectability, availability and perceptibility are treated as independent from each other.
Availability defines the probability that a bird gives a cue, and perceptibility defines the probability that an observer perceives the cue, \textit{conditional on} a bird giving a cue.
However, there is some evidence to show that this independence assumption may not necessarily hold for all species, particularly in species with high-pitched songs or species that do not sing often \citep{martin-schwarze_joint_2021}.
This lack of independence may be particularly evident when considering birds that are calling at a far distance.
Because the bird is far away, the observer may not be able to hear the bird at its first cue.
However, removal modelling is typically trying to model time to first detection, but does not account for distance in that time to first detection.
\citep{solymos_calibrating_2013} give recommendations as to how to model availability and perceptibility jointly, but still make the assumption that the parameters for each component are independent from each other.
Therefore, there is a need to explore this question of independence further, particularly if detectability estimates are to be used in correcting status and trend estimates and population estimates.

\subsection{How important are observer effects?}

\par Whether a bird is recorded during a biological survey is as much of a product of detectability as it is a product of effects related to the observer.
Observer effects are well-known in biological surveys, and many surveys will account for observer effects via a random effect in a mixed effects model.
Still, there are a few sources of error that a simple random effect may not necessarily be able to capture.
For example, observers may record birds slightly different from one another, even though they are still technically following the protocol used for the bird survey.
That is, one observer may record all the ``easy" birds to start and then focus on the more rare or difficult birds later in the survey.
When considering time-to-detection models to calculate availability, this may have the effect of biasing down cue rates for ``difficult" species, such as ones that are quiet or difficult to separate from other species.
Of course, if the observer is careful to still record the bird in the first time bin that they heard it (but not necessarily identified it), this could alleviate that bias.
It is also well-known that human observers are inconsistent in judging distance to a singing bird.
This is especially evident in forests, where attenuation and lack of spatial markers can severely inhibit a person's ability to judge distance \citep{alldredge_field_2007}.
Inaccurate distances can lead to inaccuracies in modelling effective detection radius in birds, which can cascade into incorrect estimates of population estimates.

\subsection{Can autonomous recording units help?}

\par Although the current NA-POPS database only considers data that are collected via traditional, human-led point counts, the use of autonomous recording units (ARUs) for bird monitoring is becoming more widely-used \citep{perezgranados_estimating_2021, shonfield_autonomous_2017, sugai_terrestrial_2019}.
ARUs provide the ability to survey in highly remote areas that may be otherwise less accessible for traditional human-collected data.
Additionally, ARUs provide the ability to easily monitor across multiple days in the same area, because the ARUs can be programmed to start and stop monitoring each day.
Because of this, ARUs effectively provide an archived record of the entire soundscape in a given area, which can be used to fill data gaps for many data-scarce species.

\par Despite these benefits, the use of ARUs for estimating detectability does have some challenges. 
The main challenge is being able to estimate distance to the bird, which is a crucial component of being able to estimate the perceptibility of a bird species \citep{buckland_distance_2015}.
With human monitoring, distance to the bird can be either estimated visually or determined exactly using laser rangefinders.
However, ARUs do not currently have that ability, because they are only collecting sound. 

\par There do, however, exist some new methodologies to estimate distance to a singing bird with ARUs.
One possible way of estimating distances is by using relative sound level \citep{sebastian-gonzalez_density_2018, yip_sound_2017}.
By analyzing the spectrogram of a sound recording, the distance to the bird can be estimated by accounting for the loudness of the sound, calibrated against known distances.
This is because the loudness of a sound decreases with distance to the observer according to the inverse square law.
While this method can give a coarse measure of distance, it is more repeatable and accurate than humans and can be used with any ARU \citep{yip_sound_2020}.
This technique can be combined with localization techniques described in \cite{hedley_direction--arrival_2017} to determine the direction-of-arrival of a bird song.
In other words, bird songs recorded through ARUs can now have a distance estimation and location estimate attached to them, meaning that ARU counts can roughly be converted into what a traditional point count data sheet would look like, but with much more auxiliary information. 

\par The growing database of ARU data that exist on ABMI’s WildTrax website (https://wildtrax.ca) provides a unique opportunity for NA-POPS to expand the scope of species detectability by making use of these freely available data.
One previous study laid the groundwork in detectability estimates using ARUs by generating an offset term that offset detectability estimates generated from human point counts \citep{van_wilgenburg_paired_2017}.
However, by applying sound pressure level and localization techniques to ARU counts that provide that data, the ARU counts can be converted into an estimate of a traditional point count, with time to first detection and estimated distance attached to each bird in the point count.
These are exactly the ancillary data needed to apply the QPAD methodology to the data to estimate additional detectability estimates. 

\par One promising pipeline to accomplish this is by making use of openly available tools such as \texttt{opensoundscape} \citep{lapp_opensoundscape_2023} or \texttt{locaR} \citep{becker_locar_2023} and any available localization datasets (for example, from Wildtrax).
These tools can be used to localize a singing bird within an array of ARUs, such that the distance to each of the microphones is then known.
By creating a dataset of bird vocalizations at different known distances to the recording device, a curve can be fitted that relates distance as a function of loudness of the bird (i.e., loudness = f(distance); \citet{sebastian-gonzalez_density_2018, yip_sound_2020}). 
Then, by analyzing ARU spectrograms (a visual representation of sound that plots loudness and frequency), the unknown distance to the bird can be estimated based on the bird’s loudness, by plugging in the loudness (in decibels) to the curve that was previously fit for that species (i.e., distance = f-1(loudness)). 
Previous work by \citet{haupert_physicsbased_2023} provide a reasonable framework to relate the loudness to the distance, by also accounting for other environmental variables.
Preliminary work has already made use of this pipeline (see https://github.com/BrandonEdwards/soundDist) with the hopes of then translating these estimated distances into point count-like surveys, to then apply QPAD estimates to generate EDR.


\section{Concluding Remarks}
\par Detectability is a highly nuanced metric.
My thesis has focused particularly on attempting to generate broad-scale estimates of detectability (as in Chapter 3), with some emphasis on then trying to improve these estimates afterward (as in Chapter 4).
These improvements also come with the ability to predict for data-sparse species by allowing the sharing of information between data-rich and data-poor species.
With this in mind, I feel this body of research has opened up a whole body of potential research avenues to further refine these detectability estimates.
This is evident by the creation of Boreal Avian Modelling project's Detectability Working group, and by the detectability-related symposium at the 2023 American Ornithological Society's meeting.
It is easy to see why it is important to have good estimates of detectability.
When applying detectability to the North American Breeding Bird Survey data, I showed how the trend can change significantly at the fine spatial scale of the BBS routes.
However, if something were severely wrong with the detectability estimate used, it could also then influence a wrong change in trend.
Although detectability is highly nuanced, I think there is a world of interesting modelling opportunities that lie ahead, and a very interesting opportunity to improve our understanding of biological models and surveys.

\par To bring it back to the question initially posed at the beginning of this thesis: ``If a bird calls in a forest, and there is a biologist around surveying for it, does the biologist hear the bird and record it on their data sheet?"
My answer still stays the same with ``it depends"; however, I do believe we are in an exciting time in the world of statistical and quantitative ecology such that all of the factors going into the ``it depends" are becoming more clear.
We have the data, we have the statistical modelling capabilities, we have the computational resources, and we have the domain-specific expert opinion, and so all we now need is the concerted effort to push the field of detectability research ahead.

